\documentclass[10pt, openany]{book}

\usepackage{fancyhdr}
\usepackage{imakeidx}

\usepackage{amsmath}
\usepackage{amsfonts}

\usepackage{geometry}
\geometry{letterpaper}

\usepackage{fancyvrb}
\usepackage{fancybox}
%
% Rules to allow import of graphics files in EPS format
%
\usepackage{graphicx}
\DeclareGraphicsExtensions{.eps}
\DeclareGraphicsRule{.eps}{eps}{.eps}{}
%
%  Include the listings package and define Pseudo Code
%
\usepackage{listings}
%
% Front Matter
%
\title{Tiny Lisp Interpreter}
\author{Brent Seidel \\ Phoenix, AZ}
\date{ \today }
%========================================================
%%% BEGIN DOCUMENT
\begin{document}
%
% Produce the front matter
%
\frontmatter
\maketitle
\begin{center}
This document is \copyright 2020 Brent Seidel.  All rights reserved.

\paragraph{}Note that this is a draft version and not the final version for publication.
\end{center}
\tableofcontents

\mainmatter

\chapter{Introduction}
\section{What is This?}
This is a tiny Lisp interpreter written in Ada.  It is designed to proved a language that can be embedded into other programs, including running on embedded systems without an operating system.  As a result, effort has been made to remove dependencies on Ada packages that may not be available.  A primary example is \emph{Ada.Text\_IO}.  Another feature that may be missing is dynamic memory allocation.

\section{Why is This?}


\chapter{Code Examples}
\section{Lisp Code}
Here is some sample Lisp code

\lstset{language=lisp}
\begin{lstlisting}
;
;  Read an analog pin and print the value repeatedly.
;  Digital pin 10 is tied high to keep looping and tied low to exit the loop.
;
(defun monitor-analog (n)
  (pin-mode 10 0)
  (print "Connect digital pin 10 to high to continue looping or to gnd to exit")
  (new-line)
  (print "Connect analog pin " n " to the analog value to monitor")
  (new-line)
  (print "Press <return> to continue")
  (read-line)
  (dowhile (= (read-pin 10) (+ 0 1))
    (print "Analog value is " (read-analog n))
    (new-line))
  (print "Exiting")
  (new-line))
\end{lstlisting}

\section{Ada Code}
Here is some sample Ada code
\lstset{language=Ada}
\begin{lstlisting}
   procedure first_value(e : element_type;
                         car : out element_type;
                         cdr : out element_type) is
      first : element_type;
      temp : element_type;
      s : cons_index;
   begin
      if e.kind = E_NIL then
         car := NIL_ELEM;
         cdr := NIL_ELEM;
      elsif e.kind /= E_CONS then
         car := indirect_elem(e);
         cdr := NIL_ELEM;
      else -- The only other option is E_CONS
         s := e.ps;
         first := cons_table(s).car;
         cdr :=  cons_table(s).cdr;
         if first.kind = E_NIL then
            car := NIL_ELEM;
         elsif first.kind /= E_CONS then
            car := indirect_elem(first);
         else -- The first item is a E_CONS
            temp := eval_dispatch(first.ps);
            if temp.kind = E_NIL then
               car := NIL_ELEM;
            elsif temp.kind /= E_CONS then
               car := temp;
            else
               car := cons_table(temp.ps).car;
               cdr := cons_table(temp.ps).cdr;
            end if;
         end if;
      end if;
   end;

\end{lstlisting}
\end{document}


-rw-r--r--  1 brent  staff     825 Jun 25 19:48 Tiny-Lisp.aux
-rw-r--r--  1 brent  staff   14269 Jun 25 19:48 Tiny-Lisp.log
-rw-r--r--  1 brent  staff  111440 Jun 25 19:48 Tiny-Lisp.pdf
-rw-r--r--  1 brent  staff   38179 Jun 25 19:48 Tiny-Lisp.synctex.gz
-rw-r--r--@ 1 brent  staff    3289 Jun 25 19:48 Tiny-Lisp.tex
-rw-r--r--  1 brent  staff     378 Jun 25 19:48 Tiny-Lisp.toc

